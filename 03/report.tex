\documentclass{bsuir}
\usepackage{ulem}
\usepackage{makecell}
\usepackage{multirow}

\departmentlong{инженерной психологии и эргономики}
\worktitle{Практическая работа \textnumero4\\\textquote{Промышленная собственность}}
\titleleft{
    Проверил:\\
    Фомин Д.А.\\
    ~
}
\titleright{
    Выполнил:\\
    Бородин А.Н.\\
    гр. 310901
}
\titlepageyear{2025}

\usepackage{pgfplots}
\usepackage{amsmath}
\usepackage{breqn}

\newlength{\tablewidth}
\setlength{\tablewidth}{\textwidth - \parindent}
\renewcommand{\thesection}{\arabic{section}}

\begin{document}

\maketitle
\mainmatter

% blank will be here

\addtocounter{page}{2}

\section{Формула}

Упаковщик пультов радиоуправления в защищённые капсулы для перевозки, содержащий транспортный механизм подачи пультов,
систему ориентации пультов, модуль захвата и капсулу с фиксирующим элементом, отличающийся тем, что транспортный
механизм выполнен в виде конвейерной ленты с направляющими, система ориентации пультов включает вибрационный питатель и
оптический сенсор, а модуль захвата обеспечивает синхронизированную подачу пультов в капсулы, выполненные из
многослойного ударопрочного полимера с герметизирующим замком.

\section{Реферат}

Полезная модель относится к области упаковки электронных устройств, в частности пультов радиоуправления, и предназначена
для автоматизированной упаковки этих устройств в защитные капсулы при их транспортировке. Устройство включает
конвейерную подачу, систему ориентации с вибрационным питателем и сенсорами, а также механизм захвата, подающий пульты в
ударопрочные капсулы. Технический результат состоит в повышении надёжности защиты пультов при транспортировке и снижении
трудозатрат за счёт автоматизации процесса упаковки.

\section{Описание}

\subsection{МПК} МПК "--- B65B 53/00

\subsection*{Область техники}

Полезная модель относится к автоматизированным упаковочным системам, предназначенным для обеспечения безопасной
транспортировки электронных устройств. Применение устройства возможно в производствах, где требуется бережная упаковка
малогабаритных элементов техники, таких как пульты радиоуправления.

\subsection*{Уровень техники}

При проведении патентных исследований были выявлены следующие пять патентов"=аналогов, недостатки которых обуславливают
необходимость создания предлагаемой полезной модели:

\begin{enumerate}
    \item \href{http://patents.google.com/patent/RU2503596C2}{RU 2503596 C2} --- упаковочная машина с кассетной
    загрузкой, сложная в обслуживании.
    \item \href{http://patents.google.com/patent/US11878829B2}{US 11878829 B2} --- упаковочное устройство с независимыми
    движущимися узлами, не обеспечивающее должной герметичности.
    \item \href{http://patents.google.com/patent/JP6051299B2}{JP 6051299 B2} --- автоматическая система упаковки,
    требующая точной настройки и дорогостоящего сервиса.
    \item \href{http://patents.google.com/patent/DE102013009229B4}{DE 102013009229 B4} --- упаковочный механизм с
    высоким энергопотреблением.
    \item \href{http://patents.google.com/patent/CN114393610B}{CN 114393610 B} --- устройство, не адаптированное под
    компактные объекты управления (пульты), с громоздкой конструкцией.
\end{enumerate}

\subsection*{Сущность полезной модели}

Предлагаемое устройство решает поставленную задачу посредством применения следующих элементов:
\begin{itemize}
    \item \textit{Транспортный механизм подачи пультов:} выполнен в виде конвейера с направляющими, обеспечивающими
    стабильное перемещение пультов.
    \item \textit{Система ориентации пультов:} включает вибрационный питатель и оптический сенсор, позволяющие
    корректировать положение пультов для дальнейшей упаковки.
    \item \textit{Модуль захвата:} синхронизированный механизм, подающий пульты в капсулы.
    \item \textit{Капсулы:} изготовлены из многослойного ударопрочного полимера с герметизирующим замком, что
    обеспечивает защиту при транспортировке.
\end{itemize}

Комплексное применение указанных узлов обеспечивает надёжную упаковку каждого пульта в индивидуальную капсулу, устраняя
недостатки аналогов, а именно проблемы с герметичностью, сложностью обслуживания и неэффективностью конструкции.

\subsection*{Возможность осуществления}

Все элементы предлагаемой модели могут быть реализованы с использованием серийных комплектующих: стандартных конвейерных
систем, промышленных вибропитателей, оптических сенсоров, а также капсул, изготовленных методом литья под давлением.
Работоспособность устройства подтверждена опытом макетной сборки.

\subsection*{Источники информации}

\begin{enumerate}
    \item RU 2503596 C2 --- \url{http://patents.google.com/patent/RU2503596C2}
    \item US 11878829 B2 --- \url{http://patents.google.com/patent/US11878829B2}
    \item JP 6051299 B2 --- \url{http://patents.google.com/patent/JP6051299B2}
    \item DE 102013009229 B4 --- \url{http://patents.google.com/patent/DE102013009229B4}
    \item CN 114393610 B --- \url{http://patents.google.com/patent/CN114393610B}
\end{enumerate}

\end{document}
