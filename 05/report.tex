\documentclass{bsuir}
\usepackage{ulem}
\usepackage{makecell}
\usepackage{multirow}

\departmentlong{инженерной психологии и эргономики}
\worktitle{Практическая работа \textnumero5\\\textquote{Защита прав авторов и
правообладателей. Разрешение споров о нарушении прав в области интеллектуальной
собственности}}
\titleleft{
    Проверил:\\
    Фомин Д.А.\\
    ~
}
\titleright{
    Выполнил:\\
    Бородин А.Н.\\
    гр. 310901
}
\titlepageyear{2025}

\usepackage{pgfplots}
\usepackage{amsmath}
\usepackage{breqn}
\renewcommand{\thesection}{\arabic{section}}

\newlength{\tablewidth}
\setlength{\tablewidth}{\textwidth - \parindent}

\begin{document}

\maketitle
\mainmatter

\textbf{Цель работы}: ознакомиться с законодательством сферы защиты
интеллектуальной собственности.

\textbf{Вариант}: 2 (1 и 16).

\section{Способы защиты прав интеллектуальной собственности в Республике Беларусь}

В РБ для защиты прав интеллектуальной собственности применяются следующие
механизмы, которые, как правило, могут применяться совокупно или по отдельности.

\subsection*{Патентование}

Для охраны технических решений (изобретений, полезных моделей, промышленных
образцов) проводится патентование. Патент выдается при соответствии заявленной
технологии требованиям новизны, изобретательского уровня (для изобретений) и
промышленной применимости.

Данный способ обеспечивает исключительные права на использование, производство и
продажу объекта промышленной собственности.

\subsection*{Регистрация товарных знаков и знаков обслуживания}

Защита знаков (логотипов, названий брендов, фирменных знаков) осуществляется
посредством их государственной регистрации. Получение свидетельства о
регистрации подтверждает правообладателю исключительное право использовать
данный знак в коммерческой деятельности, запрещая его использование третьим
лицам.

\subsection*{Охрана авторских прав}

Авторские права защищают произведения науки, литературы, искусства и программы
для ЭВМ. Такая защита возникает в момент создания произведения и может быть
дополнительно задокументирована регистрацией (по желанию правообладателя) для
упрощения доказательства авторства в случае спора.

\subsection*{Защита сведений о ноу-хау (коммерческая тайна)}

Для охраны информации, представляющей коммерческую ценность, которая не является
общедоступной, применяется режим коммерческой тайны. Здесь важны договорные
отношения (например, соглашения о неразглашении информации).

\subsection*{Лицензионные и иные договоры (уступка прав, франчайзинг)}

Правообладатель может передавать свои права третьим лицам по договорам
лицензионного характера или договорам уступки прав. Такие сделки оформляются
письменным соглашением, которое определяет условия использования объекта ИС,
срок, территорию и размер компенсации.

\subsection*{Судебная защита и применение мер административного принуждения}

В случае нарушения прав интеллектуальной собственности предусмотрены возможности
обращения в суд для защиты интересов (иск о прекращении нарушения, взыскании
компенсации, предписании мер по ликвидации последствий нарушения). Помимо
судебных мер предусмотрены административные санкции и меры, предусмотренные
таможенным законодательством для защиты объектов ИС от импорта контрафактной
продукции.

\section{Использование товарного знака иностранной организации на территории Беларуси}

Согласно требованиям белорусского законодательства об интеллектуальной
собственности, правообладатель (в данном случае иностранная организация) должен
предоставить организации лицензию на использование товарного знака. Эта лицензия
оформляется в виде письменного соглашения (лицензионного договора), где
указываются:

\begin{itemize}
    \item
    полномочия, передаваемые лицензиату;
    \item
    сфера использования (например, использование знака для вывески на здании магазина);
    \item
    срок, территория и условия использования товарного знака;
    \item
    размер вознаграждения (если предусмотрено) и порядок его выплаты;
    \item
    ответственность сторон за нарушение условий договора.
\end{itemize}

Простого письма, в котором правообладатель заявляет, что не возражает против
использования товарного знака, обычно недостаточно. Такое письмо может не
содержать всех существенных условий (ограничений, сроков, объема прав),
необходимых для правомерного использования знака в коммерческих целях. Кроме
того, отсутствие формальностей может затруднить доказательство наличия
разрешения в случае возникновения споров.

Более того, в ряде случаев требуется нотариальное удостоверение или регистрация
лицензионного соглашения в соответствующих органах для дополнительного
подтверждения законности передачи прав на использование товарного знака.

Таким образом, для использования товарного знака иностранной организации на
вывеске магазина необходимо заключить полноценное лицензионное соглашение (или
аналогичный договор), в котором будут подробно регламентированы права и
обязанности сторон, а не ограничиваться просто письмом о не возражениях.

\end{document}
